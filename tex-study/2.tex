% Документ создан на базе шаблона из курса https://www.coursera.org/learn/latex

\documentclass[a4paper,12pt]{article}                       % (обязательно) начало файла

\usepackage{cmap}					                        % поиск в PDF
\usepackage[T2A]{fontenc}			                        % кодировка
\usepackage[utf8]{inputenc}			                        % кодировка исходного текста
\usepackage[english,russian]{babel}	                        % локализация и переносы

% Дополнительно для математики
\usepackage{amsmath,amsfonts,amssymb,amsthm,mathtools}      % AMS расширения от Американского математического общества
\usepackage{mathtext}                                       % русские буквы в формулах
\usepackage{icomma}                                         % умная запятая - $0,2$ число, $0, 2$ перечисление
\usepackage{euscript}                                       % шрифт Эвклид
\usepackage{mathrsfs}                                       % красивый мат. шрифт
\usepackage{leqno}                                          % нумерация формул слева
\mathtoolsset{showonlyrefs=true}                            % показывать номера только у тех формул, на которые есть \eqref{} в тексте

\title{2 Математика в \LaTeX{}}
\author{Koutsenko Dmitry}
\date{\today}

\begin{document}                                            % (обязательно) начало документа

\maketitle

Привет, \LaTeX{}!

- десятичная дробь: $0,2$

- перечисление: $0, 2$

- обычная формула: $2 + 2 = 4$

- формула в строке (авт. перенос): йцукенг1йцукенг2йцукенг3йцукенг4йцукенг5йцукенг6йцукенг7йцукенг8$1+2+3+4+5+6$йцукенг9

- выключная формула: \[2 + 2 = 4\]

- выключная формула с меткой

\begin{equation}\label{eq:mrmc}
    MR=MC
\end{equation}

А это какой-то текст со ссылкой на формулу \eqref{eq:mrmc} на странице \pageref{eq:mrmc}, которая выражает какую-то взаимосвязь.

\end{document}                                              % (обязательно) конец документа
