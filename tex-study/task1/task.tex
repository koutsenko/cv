\documentclass[a4paper,12pt]{article}

% Preamble
\usepackage[utf8]{inputenc}
\usepackage[english,russian]{babel}
\usepackage{indentfirst}
% Preamble

\begin{document}

\section*{Задание}

Помогите Эйлеру набрать в LaTeX небольшую часть главы из его труда «Новая теория движения Луны» (Эйлер Л. Новая теория движения Луны. Перевод с латинского акад. А. Н. Крылова. М.-Л.: изд-во АН СССР, 1937. 248 с.). Начните набирать текст с § 2 (со слов «Пусть фиг. 21 представляет положения…») и продолжайте до конца приведенного отрывка. Исходный текст - в файле euler.pdf.
\smallskip

\subsection*{Требования к набору}

\begin{enumerate}
    \item Обозначения, приведенные на с. 152 после слов «Делаем следующие обозначения», представьте в виде таблицы, имеющей номер и заголовок.
    \item Обращайте внимание на особенности набора математических формул, их нумерацию и расположение на строке.
    \item Вставьте в документ следующие готовые рисунки: 21, 22, 23. Сделайте так, чтобы один из этих рисунков обтекался текстом справа.
    \item Рисунки должны нумероваться автоматически. Первый рисунок должен иметь номер 1. Вместо сокращения «фиг.» везде в документе должно быть сокращение «рис.».
    \item Создавая подписи к рисункам, используйте команду caption с пустым обязательным аргументом + подключите в преамбуле пакет caption.
    \item Рисунки должны быть примерно такого же размера, что и рисунки в оригинальном тексте. Исключение может составлять рисунок, обтекаемый текстом, — его можно сделать меньшего размера.
    \item Там, где в тексте документа встречается ссылка на какой-либо объект (рисунок или формулу), номер объекта должен подставляться автоматически.
    \item В конце документа сделайте автоматически генерируемый список таблиц и автоматически генерируемый список иллюстраций.
    \item В этом практическом задании не нужно заботиться о полях, шрифте, нумерации страниц и о том, как слова переносятся в строках, главное — точное воспроизведение текста и формул.
\end{enumerate}

\subsection*{Как «сдать» работу?}

В поле для ответа укажите ссылку на ваш проект с этим заданием в онлайн-сервисе Overleaf или ShareLaTeX (вы можете либо изначально делать проект в одном из этих сервисов, либо сделать его в любом удобном вам редакторе, а потом загрузить в Overleaf или ShareLaTeX):

\begin{itemize}
    \item В Overleaf: нажмите Share, скопируйте Read Only Link и вставьте в поле для ответа.
    \item В ShareLaTeX: Share - Make Public - Allow public read only access - Make Public - Done. Скопируйте ссылку на ваш проект из адресной строки браузера и вставьте в поле для ответа.
\end{itemize}

Готовый проект должен содержать tex-файл, три рисунка и компилироваться без ошибок. Если вы не уверены, что можете выполнить работу идеально, ничего страшного: сделайте настолько близко к идеалу, насколько умеете. При частичном успехе вам будет выставлен частичный балл.

\subsection*{Критерии ревью}

\noindent 
Работы будут оцениваться путем ответа на следующие вопросы:

\begin{itemize}
    \item Указана ли в поле для ответа ссылка на проект с этим практическим заданием в Overleaf или ShareLaTeX?
    \item Компилируется ли в проекте tex-файл?
    \item Набран ли весь необходимый текст?
    \item Вставлены ли в документ три предложенные в задании рисунка?
    \item Представлены ли обозначения, приведенные на с. 152 оригинального текста, в виде таблицы, имеющей номер и заголовок?
    \item Насколько точно воспроизведены математические формулы?
    \item Имеют ли две формулы со с. 153 оригинального текста единый номер (1), тогда как все остальные формулы номеров не имеют?
    \item Обтекается ли справа текстом один из рисунков?
    \item Рисунки нумеруются автоматически, и первый рисунок имеет номер 1?
    \item Два рисунка имеют примерно такие же размеры, что и рисунки в оригинальном тексте?
    \item Подставляются ли автоматически текущие номера объектов (формул и рисунков) там, где в тексте встречаются ссылки на эти объекты?
    \item Создан ли в конце документа автоматически генерируемый список таблиц?
    \item Создан ли в конце документа автоматически генерируемый список иллюстраций?
\end{itemize}

\end{document}