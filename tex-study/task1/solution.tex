\documentclass[a4paper,12pt]{article}

% Preamble
\usepackage[utf8]{inputenc}
\usepackage[english,russian]{babel}
\usepackage{indentfirst}
% Preamble

\begin{document}

\noindent 
empty 

Пусть фиг. 21 представляет положения Солнца S, Земли T и Луны L, и пусть (.) есть центр тяжести Земли и Луны. Делаем следующие обнозначения:

Масса Солнца S
? Земли T
? Луны L

Расстояние:
S(.) = p1, ST = p2, SL = p2, TL = r

Тогда будет:
T(.) = r2 = L / T+L * r
L(.) = r2 = T / T+L r

Составим теперь выражения ускорений, которые эти тела сообщают друг друг.
(Фиг 21)

Солнце S сообщает ускорения:
Земле: f * S/p12 по направлению TS
Луне: f * S/p22 > > LS

Вследствие чего точка (.) имеет ускорения:
T/T+L * f (S/q12) по направлению, параллельному TS
L/T+L * f (S/q22) > > > LS

Ускорения Солнца, происходящие от притяжения Земли и Луны, соответственно, суть:

f * (T/p12) по направлению ST
f * L/p22 > > SL

Поэтому ускорения точки (.) относительно точки S будут:

w1= f (S+T+L)/T+L * T/f12 по направлению параллельно TS

w2= f (S+T+L/T+L) * T/f22 > > > LS.

Разлагая эти ускорения, соответственно, по направлениям (.)S и (.)L, получим, как легко видеть из подобия показанных на фиг. 22 и 23 треугольников:
w11 = w1 * f/p2 по направлению (.)S
w1" = w1 * p1/f1 > > (.)L
w2' = w2 * f/f2  > > (.)L
W'2 = w2 * f/f1  > > (.)S
w2" = w2 * f2/p2 > > L(.)

Фиг. 22.

Получим для ускорений точки (.) слагающие
W1 = w'2 + w'2 = f * (S+T+L)/T+L [ X * f/f3 + L]


\end{document}