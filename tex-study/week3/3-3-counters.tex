% Документ создан на базе шаблона из курса https://www.coursera.org/learn/latex

\documentclass[a4paper,12pt]{article}                       % (обязательно) начало файла

\usepackage{cmap}					                        % поиск в PDF
\usepackage[T2A]{fontenc}			                        % кодировка
\usepackage[utf8]{inputenc}			                        % кодировка исходного текста
\usepackage[english,russian]{babel}	                        % локализация и переносы
\usepackage{indentfirst}                                    % отступ первой строки (под)раздела / (sub)section
\usepackage{amsmath,amsfonts,amssymb,amsthm,mathtools}      % AMS

%%% Теоремы
\theoremstyle{plain}                                        % стиль по-умолчанию, переопределять необязательно
\newtheorem{theorem}{Теорема}[section]
\newtheorem{proposition}[theorem]{Утверждение}

\theoremstyle{definition}                                   % "Определение"
\newtheorem{corollary}{Следствие}[theorem]
\newtheorem{problem}{Задача}[section]

\theoremstyle{remark}                                       % "Примечание"
\newtheorem*{nonum}{Решение}

\title{3-1 Объект "теорема"}
\author{Koutsenko Dmitry}
\date{\today}

\begin{document}                                            % (обязательно) начало документа

\maketitle

\section{Теоремы}

\begin{theorem}[Простое равенство]\label{theorem1}
    $2+2=4$
\end{theorem}

Смотри теорему \ref{theorem1} на стр. \pageref{theorem1}

\begin{proposition}
    $3\times 3 = 9$
\end{proposition}

\begin{corollary}
    9 / 3 = 3
\end{corollary}

\begin{nonum}
    Всё что угодно.
\end{nonum}


\section{Новые команды}

\newcommand{\nw}{\ensuremath{\succcurlyeq}}

\begin{equation}\label{pref}
    x \nw y
\end{equation}

Предпочтение \nw\ является полным.

\newcommand{\str}[1]{%
    на стр. \pageref{#1}%
}

См. уравнение \eqref{pref} \str{pref}.

\newcommand{\qwerty}[2][X]{%
    \begin{equation}
        #1 \nw #2
    \end{equation}
}

\qwerty{Y}

$x \ge y$

\renewcommand{\ge}{\geqslant}

$x \ge y$

\section{Счетчики}

\newcounter{nc}[section]

\arabic{nc}

\setcounter{nc}{10}

\arabic{nc}

\roman{nc}

\Roman{nc}

\setcounter{nc}{532}

\Roman{nc}

\setcounter{nc}{5}

\alph{nc}

\asbuk{nc}

\Asbuk{nc}

\roman{section}

\arabic{section}

\renewcommand{\thesection}{\Asbuk{section}}

\section{Тест}

\setcounter{section}{0}

\section{Тест}

\newcommand{\z}[1]{%

\addtocounter{nc}{1}
Задача \thesection.\arabic{nc}.#1%
}

\z{Текст задачи.}

\z{Problem 2}

\end{document}                                              % (обязательно) конец документа