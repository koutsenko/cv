% \documentclass[a4paper,14pt,twocolumn]{article}       % (обязательно) начало файла
\documentclass[a4paper,14pt]{article}       % (обязательно) начало файла

\usepackage{multicol}                       % многоколоночный режим в фрагменте документа
\usepackage{cmap}					        % поиск в PDF
\usepackage[T2A]{fontenc}			        % кодировка
\usepackage[utf8]{inputenc}			        % кодировка исходного текста
\usepackage[english,russian]{babel}	        % локализация и переносы
\usepackage{indentfirst}                    % отступ первой строки (под)раздела / (sub)section
\usepackage{extsizes}                       % поддержка любых размеров шрифта
\usepackage{geometry}                       % отступы от края экрана
    \geometry{top=25mm}
    \geometry{bottom=35mm}
    \geometry{left=35mm}
    \geometry{right=25mm}
\usepackage{fancyhdr}                       % колонтитулы документа
    \pagestyle{fancy}                       % включать или нет колонтитул
    \renewcommand{\headrulewidth}{0mm}      % скрыть разделитель верхнего колонтитула
    \lfoot{Нижний левый}
    \rfoot{Нижний правый}
    \lhead{Верхний левый}
    \rhead{Верхний правый}
\usepackage{setspace}                       % межстрочное расстояние
    % \onehalfspacing                         % 1.5
    % \doublespacing                          % 2
    \singlespacing                          % 1
\usepackage{lastpage}                       % для счетчика страниц \pageref{LastPage}
\usepackage{soul}                           % модификации начертания символов
\usepackage{hyperref}                       % гиперссылки и свойства PDF документа
\hypersetup{
    unicode=true,                           % русские буквы в разделе PDF
    pdftitle={Заголовок},                   % заголовок
    pdfauthor={Автор},                      % автор
    pdfsubject={Тема},                      % тема
    pdfcreator={Создатель},                 % создатель
    pdfproducer={Производитель},            % производитель
    pdfkeywords={keyword1}{key2}{key3},     % ключевые слова
    colorlinks=true,                        % false: ссылки в рамках, true: цветные ссылки
    linkcolor=red,                          % внутренние ссылки
    citecolor=green,                        % на библиографию
    filecolor=magenta,                      % на файлы
    urlcolor=cyan                           % на URL
}
\renewcommand{\familydefault}{\sfdefault}   % начертание шрифта - без засечек
\usepackage{multicol}                       % несколько колонок

\begin{document}

Проверка

% soul package demo.
\so{letterspacing}
\caps{CAPITALS, Small Capitals}
\ul{underlining}
\st{overstriking}
\hl{highlighting}

\newpage
\pagestyle{empty}                           % выключили колонтитул для этой страницы

Новая страница


\section{Кегль}

\begin{table}[h!]
    \caption{Размеры шрифта}
    \centering 
    \begin{tabular}{|c|c|}
    \hline  \verb|\tiny|            & \tiny             крошечный           \\
    \hline  \verb|\scriptsize|      & \scriptsize       очень маленький     \\
    \hline  \verb|\footnotesize|    & \footnotesize     довольно маленький  \\
    \hline  \verb|\small|           & \small            маленький           \\
    \hline  \verb|\normalsize|      & \normalsize       нормальный          \\ % Размер указан в documentclass
    \hline  \verb|\large|           & \large            большой             \\ % +5pt
    \hline  \verb|\Large|           & \Large            еще больше          \\ % +5pt
    \hline  \verb|\LARGE|           & \LARGE            очень большой       \\ % ...
    \hline  \verb|\huge|            & \huge             огромный            \\
    \hline  \verb|\Huge|            & \Huge             громадный           \\
    \hline
    \end{tabular}
\end{table}

Способ 1 - какой-нибудь \Large обычный \normalsize текст.

\begin{small}
    Способ 2 - какой-нибудь обычный текст.
\end{small}

Способ 3 - какой-{\Large нибудь} обычный текст.

\textbf{Какой-{\Large нибудь} \textit{обычный} текст.}

Еще \emph{очень \emph{хороший} очень} текст

Еще \emph{очень \textit{хороший} очень} текст

\newpage

Счетчик последней страницы \pageref{LastPage}

                                            % начертание шрифта - вернем засечки
\renewcommand{\familydefault}{cmtt}\normalfont 
                                            

\section{Титульный лист}
\newpage
\thispagestyle{empty}                       % выключили колонтитул именно для этой страницы

\begin{center}
    \textit{Федеральное государственное автономное учреждение \\
    высшего профессионального образования}
    \vspace{0.5ex}

    \textbf{НАЦИОНАЛЬНЫЙ ИССЛЕДОВАТЕЛЬСКИЙ УНИВЕРСИТЕТ \\ "ВЫСШАЯ ШКОЛА ЭКОНОМИКИ"}
\end{center}

\begin{flushright}
    \noindent
    \textit{Фамилия Имя Отчество}
    \\
    \textit{студент факультета экономики \\(группа 211И)}
\end{flushright}

\begin{center}
    \vspace{13ex}
    \textbf{Р\,Е\,Ф\,E\,Р\,А\,T}
    \vspace{1ex}

    по какой-то дисциплине

    на тему

    \textbf{\textit{<<Заголовок>>}}
    \vfill                                  % пробелы до конца нижнего края
    Москва 2014
\end{center}

\newpage

\section{Гиперссылки}

\url{http://hse.ru}

\href{http://hse.ru}{Сайт} ВШЭ

\section{Перечни}

\begin{enumerate}
    \item Первый пункт
    \item Второй пункт
    \begin{itemize}
        \item Вложенный список
        \item Вложенный список
    \end{itemize}
    \item Третий пункт
\end{enumerate}

Еще один список
\begin{itemize}
    \item[*] Итем 1
    \item[*] Итем 2
\end{itemize}

\section{Набор текста в несколько колонок}

\begin{enumerate}
    \begin{multicols}{3}
        \item $(1+x_1)(1+x_2)^2$;
        \item $\sqrt{x_1}$;
        \item $x_1+2x_2-10$;
        \item $(0{,}5x_1+x_2)^2$;
        \item $x_2$;
        \item $\sqrt{x_1}+\sqrt{2x_2}$;
        \item $\ln(1+x_1)+2\ln(1+x_2)$;
        \item $5x_1$;
        \item $10-x_1+2x_2$;
    \end{multicols}
\end{enumerate}

\end{document}

