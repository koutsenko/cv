% Документ создан на базе шаблона из курса https://www.coursera.org/learn/latex

\documentclass[a4paper,12pt]{article}                       % (обязательно) начало файла

\usepackage{cmap}					                        % поиск в PDF
\usepackage[T2A]{fontenc}			                        % кодировка
\usepackage[utf8]{inputenc}			                        % кодировка исходного текста
\usepackage[english,russian]{babel}	                        % локализация и переносы

% Дополнительно для математики
\usepackage{amsmath,amsfonts,amssymb,amsthm,mathtools}      % AMS расширения от Американского математического общества
\usepackage{mathtext}                                       % русские буквы в формулах
\usepackage{icomma}                                         % умная запятая - $0,2$ число, $0, 2$ перечисление
\usepackage{euscript}                                       % шрифт Эвклид
\usepackage{mathrsfs}                                       % красивый мат. шрифт
\usepackage{leqno}                                          % нумерация формул слева
\mathtoolsset{showonlyrefs=true}                            % показывать номера только у тех формул, на которые есть \eqref{} в тексте

% Исправляем отсутствие отступа у первой строки (под)раздела / (sub)section
\usepackage{indentfirst}

\title{3 Математика в \LaTeX{} - формулы}
\author{Koutsenko Dmitry}
\date{\today}

\begin{document}                                            % (обязательно) начало документа

\maketitle

\section{Нюансы работы с формулами}

\subsection{Дроби}

Инлайн (мелкая дробь) $\frac{3}{6} = 0,5$

С выключкой \[\frac{3}{6} = 0,5\]

С выключкой и дополнительным этажом дроби (мелкий) \[\frac{1+\frac{4}{2}}{6} = 0,5\]

С выключкой и дополнительным этажом дроби (обычный) \[\frac{1+\dfrac{4}{2}}{6} = 0,5\]

\subsection{Скобки}

Автовысота скобок \[ (2+3) \times 5 = 25\]

Автовысота скобок (не очень) \[ (2+\frac{9}{3}) \times 5 = 25\]

Ручная высота скобок \[ \left(2+\frac{9}{3}\right) \times 5 = 25\]

Квадратные скобки \[ [2+3] \]

Фигурные скобки (экранировать слэшом) \[ \{2+3\} \]

\end{document}                                              % (обязательно) конец документа
