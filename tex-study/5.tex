% Документ создан на базе шаблона из курса https://www.coursera.org/learn/latex

\documentclass[a4paper,12pt]{article}                       % (обязательно) начало файла

\usepackage{cmap}					                        % поиск в PDF
\usepackage[T2A]{fontenc}			                        % кодировка
\usepackage[utf8]{inputenc}			                        % кодировка исходного текста
\usepackage[english,russian]{babel}	                        % локализация и переносы

% Дополнительно для математики
\usepackage{amsmath,amsfonts,amssymb,amsthm,mathtools}      % AMS расширения от Американского математического общества
\usepackage{mathtext}                                       % русские буквы в формулах
\usepackage{icomma}                                         % умная запятая - $0,2$ число, $0, 2$ перечисление
\usepackage{euscript}                                       % шрифт Эвклид
\usepackage{mathrsfs}                                       % красивый мат. шрифт
\usepackage{leqno}                                          % нумерация формул слева
% \mathtoolsset{showonlyrefs=true}                            % показывать номера только у тех формул, на которые есть \eqref{} в тексте

% Исправляем отсутствие отступа у первой строки (под)раздела / (sub)section
\usepackage{indentfirst}

% Своя команда
\DeclareMathOperator{\sgn}{\mathop{sgn}}

\title{5 Математика в \LaTeX{} - многострочные формулы}
\author{Koutsenko Dmitry}
\date{\today}

\begin{document}                                            % (обязательно) начало документа

\maketitle

\section{Формулы в несколько строк}

Многоточие
\[ 1 + 2 + 3 + \dots + 99 + 100 = 5050 \]

Многострочное окружение с переносами
\begin{multline}
    1 + 2 + 3 + 4 + 5 + 6 + 7 + \dots + \\
    + 16 + 17 + 18 + 19 + 20 + 21 + 22 + 23 + 24 + 25 + \dots + \\
    \dots + 99 + 100 = 5050
\end{multline}

Несколько формул
\begin{align}
    2 \times 2 = 4                      \\
    1 + 2 + 3 + \dots + 99 + 100 = 5050 \\
    10 \times 3 = 30
\end{align}

Несколько формул выравнены по знаку равно
\begin{align}
    2 \times 2                   & = 4    \\
    1 + 2 + 3 + \dots + 99 + 100 & = 5050 \\
    10 \times 3                  & = 30
\end{align}

Несколько формул выравнены по знаку равно, есть второй столбец
\begin{align}
    2 \times 2               & = 4    & 1 \times 2 & = 2 \\
    1 + 2 + \dots + 99 + 100 & = 5050 & 2 \times 2 & = 4 \\
    10 \times 3              & = 30   & 3 \times 2 & = 6
\end{align}

Несколько формул, у них единый номер
\begin{equation}
    \begin{aligned}
        2 \times 2               & = 4    & 1 \times 2 & = 2 \\
        1 + 2 + \dots + 99 + 100 & = 5050 & 2 \times 2 & = 4 \\
        10 \times 3              & = 30   & 3 \times 2 & = 6
    \end{aligned}
\end{equation}

Система уравнений, фигурная скобка только слева
\[\left\{
    \begin{aligned}
        2 \times 2               & = 4    \\
        1 + 2 + \dots + 99 + 100 & = 5050 \\
        10 \times 3              & = 30
    \end{aligned}\right. % Точка это фантомная скобка.
\]

Модуль числа (скобки - способ 2)
\[
    |x| = \begin{cases}
        x,  & \text{если } x \ge 0 \\
        -x, & \text{если } x < 0
    \end{cases}
\]

\end{document}                                              % (обязательно) конец документа
