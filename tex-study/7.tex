% Документ создан на базе шаблона из курса https://www.coursera.org/learn/latex

\documentclass[a4paper,12pt]{article}                       % (обязательно) начало файла

\usepackage{cmap}					                        % поиск в PDF
\usepackage[T2A]{fontenc}			                        % кодировка
\usepackage[utf8]{inputenc}			                        % кодировка исходного текста
\usepackage[english,russian]{babel}	                        % локализация и переносы
\usepackage{graphicx}                                       % поддержка изображений
\usepackage{array}                                          % чтобы работал extrarowheight
\usepackage{tabularx}                                       % чтобы работал tabularx
\usepackage{tabulary}                                       % чтобы работал tabulary
\graphicspath{{images/}}                                    % путь к изображениям

% Исправляем отсутствие отступа у первой строки (под)раздела / (sub)section
\usepackage{indentfirst}

\title{7 Изображения и размеры}
\author{Koutsenko Dmitry}
\date{\today}

\begin{document}                                            % (обязательно) начало документа

\maketitle

\section{Картинки}

\subsection{Вектор}

Просто картинка\\
\includegraphics{7-image.pdf}

Картинка по центру\\
\[ \includegraphics{7-image.pdf} \]

Просто картинка увеличенная в 2 раза\\
\includegraphics[scale=2]{7-image.pdf}

Просто картинка шириной 5 см\\
\includegraphics[width=5cm]{7-image.pdf}

Просто картинка, ширина 5, высота 10\\
\includegraphics[width=5cm,height=10cm]{7-image.pdf}

Картинка, ширина 2, высота 10, пропорциональная\\
\includegraphics[width=2cm,height=10cm,keepaspectratio]{7-image.pdf}

\subsection{Растр}

Просто картинка\\
\includegraphics{7-image.png}

Просто картинка шириной в текст (ширина страницы минус поля)\\
\includegraphics[width=\textwidth]{7-image.png}

Картинка шириной в страницу\\
\includegraphics[width=\paperwidth]{7-image.png}

Картинка шириной в полстраницы\\
\includegraphics[width=0.5\paperwidth]{7-image.png}

\section{Таблицы}


Таблица, колонки выравнены по центру и разделены верт. линиями

\begin{tabular}{c|c|c}
    1 & 2  & 3 \\
    2 & 4  & 5 \\
    3 & 67 & 5
\end{tabular}

Таблица, колонки выравнены по-разному

\begin{tabular}{lrc}
    1 & 2  & 3 \\
    2 & 4  & 5 \\
    3 & 67 & 5
\end{tabular}

Таблица с горизонтальной линией-разделителем

\begin{tabular}{lll}
    1 & 2  & 3 \\
    \hline
    2 & 4  & 5 \\
    3 & 67 & 5
\end{tabular}

Таблица с разделителями

\begin{tabular}{|l|l|l|}
    \hline
    1 & $2\times2=4$ & 3 \\
    \hline
    2 & 4            & 5 \\
    3 & 67           & 5 \\
    \hline
\end{tabular}

Таблица с многострочными ячейками

% Глобальный паддинг столбца сверху? 
% \setlength{\extrarowheight}{6mm}
% FIXME Не получилось применить выборочно, т.е. выставить потом обратно в 0

\setlength{\extrarowheight}{6mm}
\begin{tabular}{|p{4cm}|l|l|}
    \hline
    1450934059 30454595 83405830 9839058 430 9 & $2\times2=4$                 & 3           \\ [0mm]
    \hline
    \[ 2\times2=4 \]                           & $\frac{6}{3}=2$              & Адаптивный  \\
    \hline
    \[ 2\times2=4 \]                           & $\displaystyle\frac{6}{3}=2$ & Нормальный  \\
    \hline
    3                                          & 67                           & 20mm высота \\ [20mm]
    \hline
\end{tabular}

Таблица с автоподстройкой ширины столбца по тексту

\begin{tabularx}{\textwidth}{|X|X|X|}
    \hline
    1450934059 30454595 83405830 9839058 430 9 & Какой-то текст покороче & А вот это очень очень длинный текст текстище \\
    \hline
\end{tabularx}

Таблица с автоподстройкой ширины столбца чтобы была оптимальная высота строки

\begin{tabulary}{\textwidth}{|C|J|R|}
    \hline
    1450934059 30454595 83405830 9839058 430 9 & Какой-то текст покороче & А вот это очень очень длинный текст текстище \\
    \hline
\end{tabulary}


\end{document}                                              % (обязательно) конец документа
