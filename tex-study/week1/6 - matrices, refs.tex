% Документ создан на базе шаблона из курса https://www.coursera.org/learn/latex

\documentclass[a4paper,12pt,leqno]{article}                 % (обязательно) начало файла, номера слева

\usepackage{cmap}					                        % поиск в PDF
\usepackage[T2A]{fontenc}			                        % кодировка
\usepackage[utf8]{inputenc}			                        % кодировка исходного текста
\usepackage[english,russian]{babel}	                        % локализация и переносы

% Дополнительно для математики
\usepackage{amsmath,amsfonts,amssymb,amsthm,mathtools}      % AMS расширения от Американского математического общества
\usepackage{mathtext}                                       % русские буквы в формулах
\usepackage{icomma}                                         % умная запятая - $0,2$ число, $0, 2$ перечисление
\usepackage{euscript}                                       % шрифт Эвклид
\usepackage{mathrsfs}                                       % красивый мат. шрифт
\usepackage{leqno}                                          % нумерация формул слева
\mathtoolsset{showonlyrefs=true}                            % показывать номера только у тех формул, на которые есть \eqref{} в тексте

% Исправляем отсутствие отступа у первой строки (под)раздела / (sub)section
\usepackage{indentfirst}

% Своя команда
\DeclareMathOperator{\sgn}{\mathop{sgn}}

\title{6 Математика в \LaTeX{} - матрицы}
\author{Koutsenko Dmitry}
\date{\today}

\begin{document}                                            % (обязательно) начало документа

\maketitle

\section{Матрицы}

Матрица с круглыми скобками
\[
    \begin{pmatrix}
        a_{11} & a_{12} & a_{13} \\
        a_{21} & a_{22} & a_{23} \\
        a_{31} & a_{32} & a_{33}
    \end{pmatrix}
\]

Матрица с прямыми скобками
\[
    \begin{vmatrix}
        a_{11} & a_{12} & a_{13} \\
        a_{21} & a_{22} & a_{23} \\
        a_{31} & a_{32} & a_{33}
    \end{vmatrix}
\]

Матрица со скобками
\[
    \begin{bmatrix}
        a_{11} & a_{12} & a_{13} \\
        a_{21} & a_{22} & a_{23} \\
        a_{31} & a_{32} & a_{33}
    \end{bmatrix}
\]

Выражение с номером (скрытым из-за showonlyrefs)
\begin{align}
    1+1=2
\end{align}

Выражение без номера
\begin{align*}
    1+1=2
\end{align*}

Выражение с заданным номером (скрытым из-за showonlyrefs)
\begin{align}
    1+1=2 \tag{S1}
\end{align}

Выражение с заданным номером, на который можно сослаться
\begin{align}
    1+1=2 \tag{S2} \label{eq:sum}
\end{align}

Ссылка на \eqref{eq:sum}, в котором что-то интересное

\end{document}                                              % (обязательно) конец документа
