\documentclass[a4paper,12pt]{article}

\usepackage[utf8]{inputenc}
\usepackage[T2A]{fontenc}
\usepackage[english,russian]{babel}

% Переопределяем нумерацию по-русски для 2-го уровня списков %
% Уровень задается количеством символов "i" после enum %
\renewcommand{\theenumii}{\asbuk{enumii}}

\begin{document}

\begin{enumerate}
  \item{Первый пункт нумерованного списка}
  \item{Второй пункт нумерованного списка}
  \item{Третий пункт того же, и со вложенными}
    \begin{enumerate}
      \item{Подпункт 1}
      \item{Подпункт 2}
      \item{Подпункт n}
    \end{enumerate}
  \item{Ну и четвертый до кучи}
\end{enumerate}

\end{document}

