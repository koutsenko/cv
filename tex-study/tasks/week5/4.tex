\documentclass[a4paper,14pt]{article}

\usepackage[utf8]{inputenc}
\usepackage[T2A]{fontenc}   % Отображение кириллицы
\usepackage[english,russian]{babel}

\usepackage{extsizes}       % Возможность сделать 14-й шрифт
\usepackage{geometry}       % Простой способ задавать поля
\usepackage{setspace}       % Возможность задать межстрочный интервал
\usepackage{indentfirst}    % Авторасстановка красных строк
\usepackage{hyperref}       % Ссылки в PDF

\geometry{top=30mm}
\geometry{bottom=40mm}
\geometry{left=30mm}
\geometry{right=20mm}
\onehalfspacing
\hypersetup{
    unicode=true,           % Русские буквы в разделе PDF
    colorlinks=true,        % False: ссылки в рамках, true: цветные ссылки
    linkcolor=black,        % Внутренние ссылки
    urlcolor=blue           % На URL
}

% Декларируем ссылки
\def \lumURL {https://ru.wikipedia.org/wiki/\%D0\%9B\%D1\%8E\%D0\%BC\%D0\%B8\%D0\%BD\%D0\%B5\%D1\%81\%D1\%86\%D0\%B5\%D0\%BD\%D1\%86\%D0\%B8\%D1\%8F}
\def \atmURL {https://ru.wikipedia.org/wiki/\%D0\%90\%D1\%82\%D0\%BC\%D0\%BE\%D1\%81\%D1\%84\%D0\%B5\%D1\%80\%D0\%B0}
\def \magURL {https://ru.wikipedia.org/wiki/\%D0\%9C\%D0\%B0\%D0\%B3\%D0\%BD\%D0\%B8\%D1\%82\%D0\%BE\%D1\%81\%D1\%84\%D0\%B5\%D1\%80\%D0\%B0}
\def \sovURL {https://ru.wikipedia.org/wiki/\%D0\%A1\%D0\%BE\%D0\%BB\%D0\%BD\%D0\%B5\%D1\%87\%D0\%BD\%D1\%8B\%D0\%B9_\%D0\%B2\%D0\%B5\%D1\%82\%D0\%B5\%D1\%80}

\begin{document}

\newcommand{\stoptocwriting}{\addtocontents{toc}{\protect\setcounter{tocdepth}{-5}}}
\newcommand{\resumetocwriting}{\addtocontents{toc}{\protect\setcounter{tocdepth}{\arabic{tocdepth}}}}

\setcounter{secnumdepth}{0} % Выключить нумерацию разделов и подразделов 

\stoptocwriting
\section{Полярное сияние\textsuperscript{\ref{mynote}}}
\resumetocwriting
\textbf{Полярное сияние (северное сияние)} — свечение (\href{\lumURL}{люминесценция}) верхних слоёв \href{\atmURL}{атмосфер планет}, 
обладающих \href{\magURL}{магнитосферой}, вследствие их взаимодействия с заряженными частицами \href{\sovURL}{солнечного ветра}.

\tableofcontents

\footnote{\label{mynote}Википедия. (2015). Полярное сияние - Википедия, свободная энциклопедия. [Online; accessed 10-января-2016]. Retrieved
from \href{https://ru.wikipedia.org/?oldid=75221213}{https://ru.wikipedia.org/?oldid=75221213}}

\subsection{Природа полярных сияний}
В очень ограниченном участке верхней атмосферы сияния могут быть вызваны низкоэнергичными заряженными частицами 
солнечного ветра, попадающими в полярную ионосферу через северный и южный полярные \textbf{каспы}. В северном полушарии 
каспенные сияния можно наблюдать над Шпицбергеном в околополуденные часы. \\
\indent При столкновении энергичных частиц плазменного слоя с верхней атмосферой происходит возбуждение атомов и молекул газов,
входящих в её состав. Излучение возбуждённых атомов в видимом диапазоне и наблюдается как полярное сияние. Спектры 
полярных сияний зависят от состава атмосфер планет: так, например, если для Земли наиболее яркими являются линии излучения 
возбуждённых кислорода и азота в видимом диапазоне, то для Юпитера — линии излучения водорода в ультрафиолете. \\
\indent Поскольку ионизация заряженными частицами происходит наиболее эффективно в конце пути частицы и плотность атмосферы
падает с увеличением высоты в соответствии с барометрической формулой, то высота появлений полярных сияний достаточно
сильно зависит от параметров атмосферы планеты, так, для Земли с её достаточно сложным составом атмосферы красное
свечение кислорода наблюдается на высотах 200—400 км, а совместное свечение азота и кислорода — на высоте $\sim$110 км.
Кроме того, эти факторы обусловливают и форму полярных сияний — размытая верхняя и достаточно резкая нижняя границы.

\subsection{Полярные сияния Земли}
Полярные сияния наблюдаются преимущественно в высоких широтах обоих полушарий в овальных зонах-поясах, 
окружающих магнитные полюса Земли — авроральных овалах. Диаметр авроральных овалов составляет 3000 км во время 
спокойного Солнца, на дневной стороне граница зоны отстоит от магнитного полюса на 10—16\textdegree, 
на ночной — 20—23\textdegree. Поскольку магнитные полюса Земли отстают от географических на $\sim$12\textdegree,
полярные сияния наблюдаются в широтах 67—70\textdegree, однако во времена солнечной активности авроральный овал 
расширяется и полярные сияния могут наблюдаться в более низких широтах — на 20—25\textdegree южнее или севернее границ 
их обычного проявления. Например, на острове Стюарт, лежащем лишь на 47\textdegree параллели, сияния происходят регулярно. 
Маори даже назвали его «Пылающие небеса». \\
\indent В спектре полярных сияний Земли наиболее интенсивно излучение основных компонентов атмосферы — азота и кислорода, при этом 
наблюдаются их линии излучения как в атомарном, так и молекулярном (нейтральные молекулы и молекулярные ионы) состоянии. Самыми 
интенсивными являются линии излучения атомарного кислорода и ионизированных молекул азота. \\
\indent Свечение кислорода обусловлено излучением возбужденных атомов в метастабильных состояниях с длинами волн 557,7 нм (зелёная линия,
время жизни 0,74 с) и дублетом 630 и 636,4 нм (красная область, время жизни 110 с). Вследствие этого красный дублет излучается 
на высотах 150—400 км, где вследствие высокой разреженности атмосферы низка скорость гашения возбужденных состояний при столкновениях. 
Ионизированные молекулы азота излучают при 391,4 нм (ближний ультрафиолет), 427,8 нм (фиолетовый) и 522,8 нм (зелёный). Однако, каждое
явление обладает своей неповторимой гаммой, в силу непостоянства химического состава атмосферы и погодных факторов.\\
\indent Спектр полярных сияний меняется с высотой. В зависимости от преобладающих в спектре полярного сияния линий излучения полярные 
сияния делятся на два типа: высотные полярные сияния типа A с преобладанием атомарных линий и полярные сияния типа B на относительно 
небольших высотах (80—90 км) с преобладанием молекулярных линий в спектре вследствие столкновительного гашения атомарных возбужденных 
состояний в сравнительно плотной атмосфере на этих высотах. \\
\indent Полярные сияния весной и осенью возникают заметно чаще, чем зимой и летом. Пик частотности приходится на периоды, ближайшие к 
весеннему и осеннему равноденствиям. Во время полярного сияния за короткое время выделяется огромное количество энергии. Так, за одно из 
зарегистрированных в 2007 году возмущений выделилось $5\cdot10^{14}$ джоулей, примерно столько же, сколько во время землетрясения
 магнитудой 5,5. \\
\indent При наблюдении с поверхности Земли полярное сияние проявляется в виде общего быстро меняющегося свечения неба или движущихся лучей,
полос, корон, «занавесей». Длительность полярных сияний составляет от десятков минут до нескольких суток. Считалось, что полярные
сияния в северном и южном полушарии являются симметричными. Однако одновременное наблюдение полярного сияния в мае 2001 из космоса
со стороны северного и южного полюсов показало, что северное и южное сияние существенно отличаются друг от друга

\subsubsection{Искусственно созданные}
Полярные сияния можно создать искусственно и затем изучать. Этому был посвящён, например, советско-французский эксперимент АРАКС,
проведённый в 1975 году.

\subsection{В культуре}
\begin{itemize}
 \item[\textcolor{blue}{\textbullet}]{\textit{«Радиоволна»}, художественный фильм 2000 г., США}
 \item[\textcolor{blue}{\textbullet}]{\textit{«Белый плен»}, художественный фильм 2006 г., США}
 \item[\textcolor{blue}{\textbullet}]{\textit{«Белая мгла»}, художественный фильм 2009 г., США}
\end{itemize}

\section{Литература}
Kruesi, L. (2009). Aurorae are not mirror images. Astronomy, 37 (11).
Александров, Н. Л. (2001). Полярные сияния. Соросовский образовательный журнал, 7 (5).
Булат, В. Л. (1974). Оптические явления в природе. Просвещение.
Зверева, С. В. (1988). В мире солнечного света. Гидрометеоиздат.
Исаев, С. И. (1980). Полярные сияния. Книж. изд-во.
Мизун, Ю. Г. (1983). Полярные сияния. Наука.
Мишин, Р. et al. (1989). Взаимодействие электронных потоков с ионосферной плазмой. Гидрометеоиздат.

\end{document}

