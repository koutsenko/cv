%-----------------------------------------------------------------------------------------------------------------------------------------------%
%	The MIT License (MIT)
%
%	Copyright (c) 2015 Jan Küster
%
%	Permission is hereby granted, free of charge, to any person obtaining a copy
%	of this software and associated documentation files (the "Software"), to deal
%	in the Software without restriction, including without limitation the rights
%	to use, copy, modify, merge, publish, distribute, sublicense, and/or sell
%	copies of the Software, and to permit persons to whom the Software is
%	furnished to do so, subject to the following conditions:
%	
%	THE SOFTWARE IS PROVIDED "AS IS", WITHOUT WARRANTY OF ANY KIND, EXPRESS OR
%	IMPLIED, INCLUDING BUT NOT LIMITED TO THE WARRANTIES OF MERCHANTABILITY,
%	FITNESS FOR A PARTICULAR PURPOSE AND NONINFRINGEMENT. IN NO EVENT SHALL THE
%	AUTHORS OR COPYRIGHT HOLDERS BE LIABLE FOR ANY CLAIM, DAMAGES OR OTHER
%	LIABILITY, WHETHER IN AN ACTION OF CONTRACT, TORT OR OTHERWISE, ARISING FROM,
%	OUT OF OR IN CONNECTION WITH THE SOFTWARE OR THE USE OR OTHER DEALINGS IN
%	THE SOFTWARE.
%	
%
%-----------------------------------------------------------------------------------------------------------------------------------------------%


%============================================================================%
%
%	DOCUMENT DEFINITION
%
%============================================================================%

%we use article class because we want to fully customize the page and dont use a cv template
\documentclass[10pt,A4]{article}	


%----------------------------------------------------------------------------------------
%	ENCODING
%----------------------------------------------------------------------------------------

%we use utf8 since we want to build from any machine
\usepackage[utf8]{inputenc}		

%----------------------------------------------------------------------------------------
%	LOGIC
%----------------------------------------------------------------------------------------

% provides \isempty test
\usepackage{xifthen}

%----------------------------------------------------------------------------------------
%	FONT
%----------------------------------------------------------------------------------------

% some tex-live fonts - choose your own

% \usepackage[defaultsans]{droidsans}
\usepackage[default]{comfortaa}
% \usepackage{cmbright}
% \usepackage[default]{raleway}
% \usepackage{fetamont}
% \usepackage[default]{gillius}
% \usepackage[light,math]{iwona}
% \usepackage[thin]{roboto} 

% set font default
\renewcommand*\familydefault{\sfdefault} 	
\usepackage[T1]{fontenc}

% more font size definitions
\usepackage{moresize}

% localization
\usepackage[english,russian]{babel}


%----------------------------------------------------------------------------------------
%	PAGE LAYOUT  DEFINITIONS
%----------------------------------------------------------------------------------------

%debug page outer frames
%\usepackage{showframe}			


%define page styles using geometry
\usepackage[a4paper]{geometry}		

% for example, change the margins to 2 inches all round
\geometry{top=1.75cm, bottom=-.6cm, left=1.5cm, right=1.5cm} 	

%use customized header
\usepackage{fancyhdr}				
\pagestyle{fancy}

%less space between header and content
\setlength{\headheight}{-5pt}		


%customize entries left, center and right
\lhead{}
\chead{ \small{Фронтенд-разработчик $\cdot$ Москва, Россия $\cdot$ \textcolor{sectcol}{\textbf{koutsenko@gmail.com}} $\cdot$ 8(905)723-18-41 $\cdot$ \textcolor{sectcol}{\textbf{https://github.com/koutsenko}} }}
\rhead{}


%indentation is zero
\setlength{\parindent}{0mm}

%----------------------------------------------------------------------------------------
%	TABLE /ARRAY DEFINITIONS
%---------------------------------------------------------------------------------------- 

%for layouting tables
\usepackage{multicol}			
\usepackage{multirow}

%extended aligning of tabular cells
\usepackage{array}

\newcolumntype{x}[1]{%
>{\raggedleft\hspace{0pt}}p{#1}}%


%----------------------------------------------------------------------------------------
%	GRAPHICS DEFINITIONS
%---------------------------------------------------------------------------------------- 

%for header image
\usepackage{graphicx}

%for floating figures
\usepackage{wrapfig}
\usepackage{float}
%\floatstyle{boxed} 
%\restylefloat{figure}

%for drawing graphics		
\usepackage{tikz}				
\usetikzlibrary{shapes, backgrounds,mindmap, trees}


%----------------------------------------------------------------------------------------
%	Color DEFINITIONS
%---------------------------------------------------------------------------------------- 

\usepackage{color}

%accent color
\definecolor{sectcol}{RGB}{255,128,0}

%dark background color
\definecolor{bgcol}{RGB}{110,110,110}

%light background / accent color
\definecolor{softcol}{RGB}{225,225,225}


%============================================================================%
%
%
%	DEFINITIONS
%
%
%============================================================================%

%----------------------------------------------------------------------------------------
% 	HEADER
%----------------------------------------------------------------------------------------

% remove top header line
\renewcommand{\headrulewidth}{0pt} 

%remove botttom header line
\renewcommand{\footrulewidth}{0pt}	  	

%remove pagenum
\renewcommand{\thepage}{}	

%remove section num		
\renewcommand{\thesection}{}			

%----------------------------------------------------------------------------------------
% 	ARROW GRAPHICS in Tikz
%----------------------------------------------------------------------------------------

% a six pointed arrow poiting to the left
\newcommand{\tzlarrow}{(0,0) -- (0.2,0) -- (0.3,0.2) -- (0.2,0.4) -- (0,0.4) -- (0.1,0.2) -- cycle;}	

% include the left arrow into a tikz picture
% param1: fill color
%
\newcommand{\larrow}[1]
{\begin{tikzpicture}[scale=0.58]
	 \filldraw[fill=#1!100,draw=#1!100!black]  \tzlarrow
 \end{tikzpicture}
}

% a six pointed arrow poiting to the right
\newcommand{\tzrarrow}{ (0,0.2) -- (0.1,0) -- (0.3,0) -- (0.2,0.2) -- (0.3,0.4) -- (0.1,0.4) -- cycle;}

% include the right arrow into a tikz picture
% param1: fill color
%
\newcommand{\rarrow}
{\begin{tikzpicture}[scale=0.7]
	\filldraw[fill=sectcol!100,draw=sectcol!100!black] \tzrarrow
 \end{tikzpicture}
}



%----------------------------------------------------------------------------------------
%	custom sections
%----------------------------------------------------------------------------------------

% create a coloured box with arrow and title as cv section headline
% param 1: section title
%
\newcommand{\cvsection}[1]
{
\colorbox{sectcol}{\mystrut \makebox[1\linewidth][l]{
\larrow{bgcol} \hspace{-8pt} \larrow{bgcol} \hspace{-8pt} \larrow{bgcol} \textcolor{white}{\textbf{#1}}\hspace{4pt}
}}\\
}

%create a coloured arrow with title as cv meta section section
% param 1: meta section title
%
\newcommand{\metasection}[2]
{
\begin{tabular*}{1\textwidth}{p{2.4cm} p{11cm}}
\larrow{bgcol}	\normalsize{\textcolor{sectcol}{#1}}&#2\\[12pt]
\end{tabular*}
}

%----------------------------------------------------------------------------------------
%	 CV EVENT
%----------------------------------------------------------------------------------------

% creates a stretched box as cv entry headline followed by two paragraphs about 
% the work you did
% param 1:	event time i.e. 2014 or 2011-2014 etc.
% param 2:	event name (what did you do?)
% param 3:	institution (where did you work / study)
% param 4:	what was your position
% param 5:	some words about your contributions
%
\newcommand{\cvevent}[5]
{
\vspace{8pt}
	\begin{tabular*}{1\textwidth}{p{2.75cm} p{4.75cm} p{9cm}}
 		\textcolor{bgcol}{#1} & \textbf{\textcolor{bgcol}{#2}} & \textbf{#3}
	\end{tabular*}
\vspace{-4pt}
\textcolor{softcol}{\hrule}
\vspace{4pt}
	\begin{tabular*}{1\textwidth}{p{2.3cm} p{14.4cm}}
&		 \larrow{bgcol}  #4\\[3pt]
&		 \larrow{bgcol}  #5\\[6pt]
	\end{tabular*}

}

% creates a stretched box as 
\newcommand{\cveventmeta}[2]
{
	\mbox{\mystrut \hspace{87pt}\textit{#1}}\\
	#2
}

%----------------------------------------------------------------------------------------
% CUSTOM STRUT FOR EMPTY BOXES
%----------------------------------------- -----------------------------------------------
\newcommand{\mystrut}{\rule[-.3\baselineskip]{0pt}{\baselineskip}}

%----------------------------------------------------------------------------------------
% CUSTOM LOREM IPSUM
%----------------------------------------------------------------------------------------
\newcommand{\lorem}
{Lorem ipsum dolor sit amet, consectetur adipiscing elit. Donec a diam lectus.}



%============================================================================%
%
%
%
%	DOCUMENT CONTENT
%
%
%
%============================================================================%
\begin{document}


%use our custom fancy header definitions
\pagestyle{fancy}


%---------------------------------------------------------------------------------------
%	TITLE HEADLINE
%----------------------------------------------------------------------------------------
\vspace{-20.55pt}

% use this for multiple words like working titles etc.
%\hspace{-0.25\linewidth}\colorbox{bgcol}{\makebox[1.5\linewidth][c]{\hspace{46pt}\HUGE{\textcolor{white}{\textsc{Jan Küster}} } \textcolor{sectcol}{\rule[-1mm]{1mm}{0.9cm}} \parbox[b]{5cm}{   \large{ \textcolor{white}{{IT Consultant}}}\\
% \large{ \textcolor{white}{{Resume}}}}
%}}

% use this for single words, e.g. CV or RESUME etc.
\hspace{-0.25\linewidth}\colorbox{bgcol}{\makebox[1.5\linewidth][c]{\Huge{\textcolor{white}{\textsc{Куценко Дмитрий}}}}}


%----------------------------------------------------------------------------------------
%	HEADER IMAGE
%----------------------------------------------------------------------------------------

\begin{figure}[H]
	\begin{flushright}
		\includegraphics[width=0.15\linewidth]{myfoto.jpg}	%trimming relative to image size!
	\end{flushright}
\end{figure}

%---------------------------------------------------------------------------------------
%	QR CODE (optional)
%----------------------------------------------------------------------------------------
%\vspace{-136pt}
%\hspace{0.75\linewidth}
%\includegraphics[width=103pt]{qrcode}
%\normalsize
%\vspace{88pt}

%---------------------------------------------------------------------------------------
%	META SECTION
%----------------------------------------------------------------------------------------

\vspace{-120pt}

\metasection{Позиция:}{Фронтенд-разработчик}
\metasection{Навыки:}{Проектирование, разработка, тестирование}
\metasection{Знания:}{JavaScript/TypeScript, React.js, Webpack, Git, Docker, Python, Django}
\metasection{{Хобби}:}{Linux, Blockchain, Sci-Fi сериалы, готовка, путешествия}

%---------------------------------------------------------------------------------------
%	SUMMARAY (optional)
%----------------------------------------------------------------------------------------

%\cvsection{Summary}\\
%I am a digital media graduate (M.Sc.) with project experience in educational research as well as in the private sector. During my studies I focused on e-assessment software and moved over to b2b software for IBM Notes Domino.

%Currently I develop and evaluate the next generation learning management system with Meteor based on an extensive nursing curriculum for healthcare education.  I also love fitness, martial arts, videogames, news and Sci-Fi series.\\[-2pt]

%============================================================================%
%
%	CV SECTIONS AND EVENTS (MAIN CONTENT)
%
%============================================================================%

%---------------------------------------------------------------------------------------
%	EXPERIENCE
%----------------------------------------------------------------------------------------
\vspace{18pt}
\cvsection{Опыт работы}

\cvevent{09.2020 - наст.вр.}
{Фронтенд-разработчик}
{Сбербанк}
{Разработка UI ЛК для юр.лиц, модуль банковского сопровождения}
{TypeScript, React. Поддержка легаси (СББОЛ), целевая разработка (ЕФС БС)}

\cvevent{12.2017 - 09.2020}
{Фронтенд-разработчик}
{FireOffer.Ru}
{Разработка SPA для работы с коммерческими предложениями}
{React.js MPA, WYSIWYG конструктор, генерация PDF, интеграция с amoCRM и Битрикс24}

\cvevent{03.2019 - 08.2019}
{Веб-разработчик}
{НИУ ВШЭ}
{Разработка этапа международной олимпиады по экономике (IEO 2019)}
{Симулятор трейдера - NodeJS, Google Docs API, Websockets, React.js}

\cvevent{02.2013 - 06.2014}
{Фронтенд-разработчик}
{Гарант-Сервис Университет}
{Разработка SPA для новой версии справочно-правовой системы Гарант}
{Главная страница и страницы поиска (ExtJS). Верстка для печати. Unit-тесты Siesta}

\cvevent{11.2012 - 01.2013}
{Веб-разработчик}
{СТ-Премиум}
{Сопровождение медицинского портала doctorinfo.ru}
{Верстка, ведение и пересчет рейтинга медучреждений (CodeIgniter, HTML5, jQuery)}

% \cvevent{05.2012 - 10.2012}
% {Веб-разработчик}
% {min2win.ru}
% {Сопровождение игрового портала min2win.ru}
% {Верстка, вход через соц. сети, оптимизация времени отклика (PHP, MySQL)}

% \cvevent{12.2009 - 05.2011}
% {Системный администратор}
% {ГУ МОНИКИ им. М.Ф. Владимирского}
% {Сопровождение IT-инфраструктуры предприятия (ЛВС 500+ ПК)}
% {Разворачивание домена ActiveDirectory, раздач обновлений WSUS и антивирусной сети}

% \cvevent{06.2003 - 01.2009}
% {Инженер-программист}
% {ФГУП НИИАА им. акад. В.С. Семенихина}
% {Разработка внутренних программных комплексов}
% {Нормоконтроль (C++, MySQL) и ПО для планово-экономического отдела (Java, Oracle)}

%---------------------------------------------------------------------------------------
%	EDUCATION SECTION
%--------------------------------------------------------------------------------------
\vspace{18pt}
\cvsection{Образование}

\cvevent{2016 - наст. вр.}
{Митапы/конференции}
{}
{Веб-разработка - MoscowCSS, MoscowJS, ReactJS Moscow, ReactJS Russia}
{Разное от Яндекс, Mail.Ru, иногда финтех - Райффайзен, Альфа, Тинькофф и пр.}

\cvevent{2015 - наст. вр.}
{Онлайн курсы}
{}
{Geekbrains: Git, ООП, Python, современный Javascript (ES6+)}
{Coursera: Python (Django), \LaTeX{}}

\cvevent{2008}
{Высшее - специалист}
{РТУ МИРЭА}
{Специальность 220400 - ПО вычислительной техники и автоматизированных систем}
{Диплом - разработка АРМ администратора СУБД планово-экономического отдела}

%-------------------------------------------------------------------------------------------------
%	ARTIFICIAL FOOTER (fancy footer cannot exceed linewidth) 
%--------------------------------------------------------------------------------------------------

\null
\vspace*{\fill}
\hspace{-0.25\linewidth}\colorbox{bgcol}{\makebox[1.5\linewidth][c]{\mystrut \small \textcolor{white}{last update: 10/2021}}}

%============================================================================%
%
%
%
%	DOCUMENT END
%
%
%
%============================================================================%
\end{document}
