% Документ создан на базе шаблона из курса https://www.coursera.org/learn/latex

\documentclass[a4paper,12pt]{article}                       % (обязательно) начало файла

\usepackage{cmap}					                        % поиск в PDF
\usepackage[T2A]{fontenc}			                        % кодировка
\usepackage[utf8]{inputenc}			                        % кодировка исходного текста
\usepackage[english,russian]{babel}	                        % локализация и переносы

% Дополнительно для математики
\usepackage{amsmath,amsfonts,amssymb,amsthm,mathtools}      % AMS расширения от Американского математического общества
\usepackage{mathtext}                                       % русские буквы в формулах
\usepackage{icomma}                                         % умная запятая - $0,2$ число, $0, 2$ перечисление
\usepackage{euscript}                                       % шрифт Эвклид
\usepackage{mathrsfs}                                       % красивый мат. шрифт
\usepackage{leqno}                                          % нумерация формул слева
\mathtoolsset{showonlyrefs=true}                            % показывать номера только у тех формул, на которые есть \eqref{} в тексте

% Исправляем отсутствие отступа у первой строки (под)раздела / (sub)section
\usepackage{indentfirst}

% Своя команда
\DeclareMathOperator{\sgn}{\mathop{sgn}}

\title{4 Математика в \LaTeX{} - функции}
\author{Koutsenko Dmitry}
\date{\today}

\begin{document}                                            % (обязательно) начало документа

\maketitle

\section{Функции}

Синус \[ \sin x = 5 \]

Логарифм \[ \ln x = 5 \]

Моя команда \[ \sgn x = 5 \]

\section{Символы}

Не равно \[ 2\times 2 \ne 5 \]

Пересечение множеств \[ A \cap B \]

Объединение множеств \[ A \cup B \]

\section{Диакритические знаки}

Черта над \[ \bar x = 2 \]

Тильда \[ \tilde y = 3 \]

Линия над \[ \overline{12345xyz} \]

Широкая тильда \[ \widetilde{12345xyz} \]

\section{Буквы других алфавитов}

Тангенс альфа \[ \tg \alpha = 1 \]

Тангенс фи \[ \tg \phi = 1 \]

Заглавная фи \[ \Phi \]

Эпсилон \[ \epsilon \]

Русский эпсилон \[ \varepsilon \]

Русский фи \[ \varphi \]

% TODO как написать заглавный русский фи? Он вообще есть?

\end{document}                                              % (обязательно) конец документа
