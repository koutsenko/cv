% Этот шаблон документа разработан в 2014 году
% Данилом Фёдоровых (danil@fedorovykh.ru) 
% для использования в курсе 
% <<Документы и презентации в \LaTeX>>, записанном НИУ ВШЭ
% для Coursera.org: http://coursera.org/course/latex .
% Исходная версия шаблона --- 
% https://www.writelatex.com/coursera/latex/4.2

\documentclass[a4paper,12pt,leqno]{article}

%%% Работа с русским языком
\usepackage{cmap}					% поиск в PDF
\usepackage{mathtext} 				% русские буквы в формулах
\usepackage[T2A]{fontenc}			% кодировка
\usepackage[utf8]{inputenc}			% кодировка исходного текста
\usepackage[english,russian]{babel}	% локализация и переносы

%%% Дополнительная работа с математикой
\usepackage{amsmath,amsfonts,amssymb,amsthm,mathtools} % AMS
\usepackage{icomma} % "Умная" запятая: $0,2$ --- число, $0, 2$ --- перечисление

%% Номера формул
%\mathtoolsset{showonlyrefs=true} % Показывать номера только у тех формул, на которые есть \eqref{} в тексте.
%\usepackage{leqno} % Нумерация формул слева

%% Свои команды
\DeclareMathOperator{\sgn}{\mathop{sgn}}

%% Перенос знаков в формулах (по Львовскому)
\newcommand*{\hm}[1]{#1\nobreak\discretionary{}
{\hbox{$\mathsurround=0pt #1$}}{}}

%%% Работа с картинками
\usepackage{graphicx}  % Для вставки рисунков
\graphicspath{{images/}{images2/}}  % папки с картинками
\setlength\fboxsep{3pt} % Отступ рамки \fbox{} от рисунка
\setlength\fboxrule{1pt} % Толщина линий рамки \fbox{}
\usepackage{wrapfig} % Обтекание рисунков текстом

%%% Работа с таблицами
\usepackage{array,tabularx,tabulary,booktabs} % Дополнительная работа с таблицами
\usepackage{longtable}  % Длинные таблицы
\usepackage{multirow} % Слияние строк в таблице

%%% Теоремы
\theoremstyle{plain} % Это стиль по умолчанию, его можно не переопределять.
\newtheorem{theorem}{Теорема}[section]
\newtheorem{proposition}[theorem]{Утверждение}
 
\theoremstyle{definition} % "Определение"
\newtheorem{corollary}{Следствие}[theorem]
\newtheorem{problem}{Задача}[section]
 
\theoremstyle{remark} % "Примечание"
\newtheorem*{nonum}{Решение}

%%% Программирование
\usepackage{etoolbox} % логические операторы

%%% Страница
\usepackage{extsizes} % Возможность сделать 14-й шрифт
\usepackage{geometry} % Простой способ задавать поля
	\geometry{top=25mm}
	\geometry{bottom=35mm}
	\geometry{left=35mm}
	\geometry{right=20mm}
 %
%\usepackage{fancyhdr} % Колонтитулы
% 	\pagestyle{fancy}
 	%\renewcommand{\headrulewidth}{0pt}  % Толщина линейки, отчеркивающей верхний колонтитул
% 	\lfoot{Нижний левый}
% 	\rfoot{Нижний правый}
% 	\rhead{Верхний правый}
% 	\chead{Верхний в центре}
% 	\lhead{Верхний левый}
%	\cfoot{Нижний в центре} % По умолчанию здесь номер страницы

\usepackage{setspace} % Интерлиньяж
%\onehalfspacing % Интерлиньяж 1.5
%\doublespacing % Интерлиньяж 2
%\singlespacing % Интерлиньяж 1

\usepackage{lastpage} % Узнать, сколько всего страниц в документе.

\usepackage{soul} % Модификаторы начертания

\usepackage{hyperref}
\usepackage[usenames,dvipsnames,svgnames,table,rgb]{xcolor}
\hypersetup{				% Гиперссылки
    unicode=true,           % русские буквы в раздела PDF
    pdftitle={Заголовок},   % Заголовок
    pdfauthor={Автор},      % Автор
    pdfsubject={Тема},      % Тема
    pdfcreator={Создатель}, % Создатель
    pdfproducer={Производитель}, % Производитель
    pdfkeywords={keyword1} {key2} {key3}, % Ключевые слова
    colorlinks=true,       	% false: ссылки в рамках; true: цветные ссылки
    linkcolor=red,          % внутренние ссылки
    citecolor=black,        % на библиографию
    filecolor=magenta,      % на файлы
    urlcolor=cyan           % на URL
}

\usepackage{csquotes} % Еще инструменты для ссылок

%\usepackage[style=authoryear,maxcitenames=2,backend=biber,sorting=nty]{biblatex}  % Библиография

\usepackage{multicol} % Несколько колонок

\usepackage{tikz} % Работа с графикой
\usepackage{pgfplots}
\usepackage{pgfplotstable}

\author{\LaTeX{} в Вышке}
\title{Ti\emph{k}Z}
\date{\today}

\begin{document} % конец преамбулы, начало документа

\begin{figure}[h]
	\begin{center}
		\begin{tikzpicture}[scale=2]
			\draw[help lines] (0,0) grid (2,3);
			\draw[<->] (0,3.1) -- (0,0) -- (2.1,0);
			\draw[green,fill=yellow] (1,1) circle [radius=0.5];
			\draw[red] (0,0) -- (0,1);
			\draw[red] (0,1) -- (1,1) -- (2,3);
			\draw[domain=-1:2] plot (\x, \x*\x + 2);
			\node[below] at (1,1) {\tiny Текст $R=0,5$};
		\end{tikzpicture}
		\caption{Первая картинка}
	\end{center}
\end{figure}

% Источник: http://thetarzan.wordpress.com/2011/06/19/tikz-for-economists-a-how-to-guide-with-examples/
\begin{tikzpicture}[domain=0:3.7,thick,scale=1.5]
	\usetikzlibrary{calc}           			% Расчет координат
	\usetikzlibrary{decorations.pathreplacing}  

	% Параметры функций спроса и предложения
	\def\dint{2.1}      % Пересечение D с осью.		% \def — аналог команды \renewcommand
	\def\dslp{-0.5}     % Наклон спроса.			% \def — аналог команды \renewcommand
	\def\sint{0.3}      % Пересечение S спроса.		% \def — аналог команды \renewcommand
	\def\sslp{0.8}      % Наклон предложения.		% \def — аналог команды \renewcommand

	% Функции спроса и предложения
 	\newcommand{\demand}{\x,{\dslp*\x+\dint}}
 	\newcommand{\supply}{\x,{\sslp*\x+\sint}}

	% Координаты пересечений
    \coordinate (ints) at ({(\sint-\dint)/(\dslp-\sslp)},{(\sint-\dint)/(\dslp-\sslp)*\sslp+\sint});
    \coordinate (ep) at (0,{(\sint-\dint)/(\dslp-\sslp)*\sslp+\sint});
    \coordinate (eq) at ({(\sint-\dint)/(\dslp-\sslp)},0);
    \coordinate (dint) at (0,{\dint});
    \coordinate (sint) at (0,{\sint});

    % Спрос (D)
    \draw[thick,color=green] plot (\demand) node[right] {$D$};
	% Преложение (S)
    \draw[thick,color=purple] plot (\supply) node[right] {$S$};

	% Оси
    \draw[->] (0,0) -- (4,0) node[right] {$Q$};
    \draw[->] (0,0) -- (0,3.6) node[above] {$P$};
    
    % Пнктирные линии              
    \draw[dashed] (ints) -- (eq) node[below] {$Q^*$};                   
    \draw[dashed] (ints) -- (ep) node[left] {$P^*$};                       
\end{tikzpicture}

\begin{tikzpicture}
	\begin{axis}[xlabel={Ставка процента (\%)},ylabel={Приток капитала (млрд долл. США, ППС)},clip=false,xmin=-2, xmax=31.9, ymin=-5500000000, ymax=8500000000,axis lines=center,width=16cm,height=10cm,scale=0.8] 
		\addplot [only marks] table[x index=0,y index=1,col sep=comma] {data.dat};  
		\node[pin={[pin edge={thick}]190:{Белоруссия}}] at (axis cs:29.7,-1876683214) {};
		\node[pin={[pin edge={thick}]350:{США}}] at (axis cs:1.5,6956000000) {};
		\node[pin={[pin edge={thick}]350:{Румыния}}] at (axis cs:13,2463000000) {};
		\node[pin={[pin edge={thick}]10:{Россия}}] at (axis cs:0.6,-5217800000) {};
		\node[pin={[pin edge={thick}]260:{Великобритания}}] at (axis cs:-1.2,5983774999) {};
		\node[pin={[pin edge={thick}]272:{Армения}}] at (axis cs:18.8,108025351) {};
		\node[pin={[pin edge={thick}]350:{Швейцария}}] at (axis cs:2.6,-2044279359) {};
		\node[pin={[pin edge={thick}]10:{Китай}}] at (axis cs:4.1,4272276974) {};
		\node[pin={[pin edge={thick}]260:{Восточный Тимор}}] at (axis cs:-1.5,23415198) {};
	\end{axis}
\end{tikzpicture}

\usetikzlibrary{arrows}
\begin{tikzpicture}[line cap=round,line join=round,>=triangle 45,x=1.0cm,y=1.0cm]
	\draw[->,color=black] (0,0) -- (6.37,0);
	\foreach \x in {,1,2,3,4,5,6}
		\draw[shift={(\x,0)},color=black] (0pt,2pt) -- (0pt,-2pt) node[below] {\footnotesize $\x$};
	\draw[->,color=black] (0,0) -- (0,5.25);
	\foreach \y in {,1,2,3,4,5}
		\draw[shift={(0,\y)},color=black] (2pt,0pt) -- (-2pt,0pt) node[left] {\footnotesize $\y$};
	\draw[color=black] (0pt,-10pt) node[right] {\footnotesize $0$};
	\clip(0,0) rectangle (6.37,5.25);
	\draw (0.05,2.75)-- (2.14,0);
	\draw (0.07,0.24)-- (3.33,1.82);
	\draw(2.41,1.99) circle (2cm);
	\begin{scriptsize}
		\draw[color=black] (0.93,1.36) node {$D$};
		\draw[color=black] (1.85,0.94) node {$S$};
		\draw[color=black] (1.95,2.65) node {$c$};
	\end{scriptsize}
\end{tikzpicture}


\end{document} % конец документа

